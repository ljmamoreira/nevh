\documentclass{article}
\usepackage[utf8]{inputenc}
\usepackage[portuges]{babel}
\usepackage[T1]{fontenc}
\usepackage[a4paper,margin=2.5cm]{geometry}
\usepackage{amsmath,icomma,tikz,amsfonts}
\usepackage{bbold}

\newcommand{\pd}[2]{\frac{\partial#1}{\partial#2}}

\title{\texttt{nevh}: Numerical Evolution from the Hamiltonian}
\date{\today}
\author{LJM~Amoreira}
\begin{document}
\maketitle
\section{Introduction}
This package computes the trajectories of mechanical systems given their
hamiltonian function and initial state alone.

The partial derivatives of the hamiltonian in order to the $q$'s and $p$'s are
estimated numerically and their values used to set up and solve the time ODE.

\section{Hamilton's Equations}
The physical state of a system with $N_f$ degreees of freedom is completely
specified given the values of $2N_f$ variables $q_i$ (coordinates) and $p_i$
(momentums) $i=1,\ldots,N_f$.  Its time evolution is determined by the
Hamilton's Equations,
\begin{align}
  \dot q_i&=\pd{H}{p_i}&
  \dot p_i&=-\pd{H}{q_i},
\end{align}
where $H=H(t,q,p)$ is the hamiltonian of the system which, in many situations,
is simply the toal mechanical energy of the system, $H=T+U$. Once the partial
derivatives in the rhs of Hamilton's equations are known, they define a system
of $2N_f$ first order simultaneous ordinary differential equations (ODE) in time
for the $q$'s and the $p$'s. The solution of this system is the trajectory (in
phase space) followed by the system in its evolution.

In most cases, the above mentioned system of ODEs must be solved numerically,
and a large number of computer packages, for all significant computer languages
have been developed to do that. Here, we take this numerical approach a step
further, and compute the partial derivatives of the hamiltonian also
numerically. With this approach, the evolution of the system can be obtained
from the hamiltonian alone, saving the need for the derivation of the right hand
sides of the Hamilton's equations. 


\section{Hamilton's Equations in Numerical Form}
Hamilton's
Equations can be set in a more compact (and more suitable for numerical resolution
using standard routines or in the current approach) form,
\begin{equation}\label{eq:HEterse}
  \dot \psi=f(t,\psi),
\end{equation}
where $\psi=(\psi_1,\,\psi_2,\,\ldots,\,\psi_{2N_f})=(q_i,\,p_i)$,
$i=1,\ldots N_f$ and
\begin{equation}
  f(t,\psi)=
  \begin{pmatrix}
    \mathbb{0}&\mathbb{1}\\
    -\mathbb{1} &\mathbb{O}
  \end{pmatrix}\nabla_\psi H(t,\psi).
\end{equation}
Here, $\mathbb{0}$ and $\mathbb{1}$ represent the $N_f\times N_f$ zero matrix
and identity matrix, respectively, and
\begin{equation*}
  \nabla_\psi H=\left(
    \pd{H}{\psi_1},\,
    \pd{H}{\psi_2},\,\ldots,\,
    \pd{H}{\psi_{2N_f}}
    \right).
\end{equation*}
\end{document}
