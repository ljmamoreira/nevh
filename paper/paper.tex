\documentclass{iopart}
\usepackage[utf8]{inputenc}
\usepackage[T1]{fontenc}
\usepackage[english]{babel}
\usepackage{tikz}
\usetikzlibrary{shapes.geometric}
\usepackage{bbm}
\usepackage{listings, algorithm2e}
%\usepackage{xcolor}
\lstset{%
  language=python,
  %backgroundcolor=\color{black!5},
  basicstyle=\sffamily\footnotesize,
  keepspaces=false,
  showspaces=false,
  showstringspaces=false
}
\expandafter\let\csname equation*\endcsname\relax
\expandafter\let\csname endequation*\endcsname\relax
\usepackage{amsmath}
\newcommand{\pd}[2]{\frac{\partial#1}{\partial#2}}
\newcommand{\undertilde}[1]{\underset{\tilde{}}{#1}}
\newcommand{\hmatrix}{\ensuremath{%
  \begin{pmatrix}
    \mathbbm{O}&\mathbbm{I}\\-\mathbbm{I}&\mathbbm{O}
  \end{pmatrix}
}}
\newcommand{\numap}[1]{\ensuremath{%
    \left\lfloor#1\right\rceil
}}


\begin{document}
\title{Numerical Evolution from the Hamiltonian}
\author{José M L Amoreira$^1$ and Luís J M Amoreira$^2$}
\address{$^1$ Departamento de Física, Instituto Superior Técnico, Lisboa,
Portugal}
\address{$^2$ Departamento de Física, Universidade da Beira Interior, Covilhã,
Portugal}
\ead{amoreira@ubi.pt}
\begin{abstract}
  We propose a numerical method for approximate calculations of the time
  evolution of point particle systems given only the system's hamiltonian
  function and initial conditions. The method both generates and solves the
  equations of motion numerically. For demonstration purposes, a working
  im\-ple\-men\-ta\-tion (written in \texttt{python}) is described and applied
  to standard problems. The method may have some pedagogical merits but the
  numerical effort of generating the equations of motion makes it unsuitable for
  actual numerical solution of ``real'' problems with any but just a few degrees
  of freedom.
\end{abstract}
\noindent{\it Keywords\/}: Classical particle dynamics,
Hamilton's equations,
Numerical methods

\submitto{\EJP}
%\maketitle
\section{Introduction}
%-----
% 1 As equações de Hamilton de um sistema de massas pontuais formam um sistema
% de N ODEs de 1ª ordem, com a forma 
% \dot x_i= G_i(t, \tilde x)
%-----
In the hamiltonian formulation of classical mechanics~\cite{gol:1980, fw:2003},
the time evolution of physical systems is governed by Hamilton's equations
which, for point particle systems, take the form
\begin{align}\label{eq:heqs}
  \dot q_i&=\pd{H}{p_i}&
  \dot p_i&=-\pd{H}{q_i},&
  i&=1, 2, \ldots, N,
\end{align}
where $N$ is the number of degrees of freedom of the system (three times the
number of particles for three dimensional, unconstrained systems), $q_i$, $p_i$
are the canonical coordinates and conjugate momenta respectively, $H=H(t, q_i,
p_i)$ is the hamiltonian function of the system (for systems with time
independent potentials and constraints, the hamiltonian is simply the
mechanical energy function~\cite{fw:2003}) and dotted symbols denote their
total derivative with respect to time. 
Renaming the canonical coordinates and momenta as
\begin{equation}
  x_j = 
  \begin{cases}
    q_j&\quad\text{if }j\leq N\\
    p_{j-N}&\quad\text{if }j> N
  \end{cases}
  \qquad j=1, \ldots, 2N,
\end{equation}
Hamilton's equations take a more unified form
\begin{equation}\label{eq:std_ode}
  \dot x_j=G_j(t, x_1, x_2, \ldots, x_{2N}),\quad j=1, \ldots, 2N
\end{equation}
where
\begin{equation}
  G_j(t, x)=
  \begin{cases}\displaystyle
    \pd{H}{x_{j+N/2}} ,&\quad\text{if }j\leq N\\[1em]
    \displaystyle
    -\pd{H}{x_{j-N/2}}, &\quad\text{if }j> N.
  \end{cases}
\end{equation}
This expression can be further simplified as a matrix product:
\begin{equation}\label{eq:rhsmf}
  G_i=M_{i j}\pd{H}{x_j},\qquad
  \text{with }
  M=
  \begin{pmatrix}
    \mathbbm{O}&\mathbbm{I}\\
    -\mathbbm{I}&\mathbbm{O}
  \end{pmatrix},
\end{equation}
where $\mathbbm{O}$ and $\mathbbm{I}$ represent, respectively, the $N\times N$
zero and identity matrices.  Equation~\eqref{eq:std_ode} shows the Hamilton's
equations for a system of point particles with $N$ degrees of freedom as a
system of $2N$ first order ordinary differential equations (ODEs) on $2N$
variables $x_1, x_2, \ldots, x_{2N}$.

In all but a small handfull of very well known simple problems, this system of
ODEs must be solved numerically. All popular library routines for solving
ODEs\footnote{Like ODEPACK~\cite{odepack} for fortran, \texttt{odeint}
  \cite{odeint} in c++ boost libraries or 
  \texttt{solve\_ivp} in SciPy~\cite{scipy} for python.}
require the user to supply subprograms to compute the rhs functions $G_i$.
The ODE solver invokes these subprograms to compute estimates for the
values of the unknowns $x_i$ for arbitrary time $t$, given their values at a
particular instant $t_0$.

In this work, we propose an alternative numerical method, where the user only
supplies the hamiltonian function of the system, leaving its partial derivatives
to be computed numerically as well.  This procedure is computationally more
expensive, since many evaluations of the hamiltonian function are needed in
order to estimate its partial derivatives at any moment in time, and so it
really is not suitable for the analysis of complex systems with more than just a
few coupled degreess of freedom. However, for simple systems it is very
practical and, anyway, it is an interesting concept in its own, which we haven't
seen exposed elsewhere.

%-------------------------------------------------------------------------------
\section{Numerical procedure}
Standard numerical routines for solving systems of ODEs expect the problem to be
written in the form
\begin{equation}\label{eq:numheqs}
  \frac{d\psi_i}{dt}=G_i(t, \psi_1, \psi_2, \ldots),\quad i=1, \ldots, N,
\end{equation}
where $N$ is the number of unknown functions $\psi_i(t)$ (and the number of
equations in well posed problems) and $G_i(t, \psi_1, \psi_2, \ldots)\equiv
G_i(t,\psi_j)$ are the rhs of the differential equations, which must be coded by
the user as callable subprograms. For instance, the python standard initial
value problem solver routine has interface
\texttt{solve\_ivp(rhs, time\_interval, psi\_0)}; here, \texttt{rhs} is the name
of a phython function to compute the values of the rhs functions $G_i$.
During its execution, the ODE solver routine repeatedly invokes these
subprograms (as many times as required by the specific integration algorithm
and/or desired accuracy) in order to compute estimates for the values the
unknown functions $\psi_i$ at given values of $t$.

For Hamilton's equations, the $\psi_i$ are the coordinates and conjugate momenta
of the system, while the $G_i$ are the partial derivatives of its hamiltonian
function (reordered and eventually multiplied by -1, according
to~(\ref{eq:rhsmf})). The usual procedure is to derive analytical expressions
for the $G_i$ as functions of $t$, the coordinates and the momenta, and then to
code those expressions as subprograms to be called by the ODE solver routine.

In the method now proposed, instead, the user only has to supply code to
compute the value of the system's hamiltonian function; its partial derivatives
(the $G_i$ in~\eqref{eq:numheqs}) are estimated numerically and reordered by
another subprogram which the ODE solver calls. Note that the differentiation and
reordering of the partial derivatives can be computed by a general purpose
routine. Indeed, the gradient can even be computed also by a standard numerical
library function, if available for the chosen computing system or language.
Figure~\ref{fig:a} displays flowcharts for both methods.
\begin{figure}[htb]
  \centering
    \begin{tikzpicture}[node distance=1.0cm, font=\sffamily,scale=0.7]
      \footnotesize
      \tikzstyle{io} = [trapezium,
                        trapezium left angle=70,
                        trapezium right angle=110,
                        minimum width=1.0cm,
                        minimum height=.6cm,
                        text centered,
                        draw=black,
                        fill=blue!20]
      \tikzstyle{libcode} = [rectangle,
                             rounded corners,
                             minimum width=1.5cm,
                             text width=1.5cm,
                             minimum height=0.6cm,
                             text centered,
                             draw=black,
                             fill=orange!20]
      \tikzstyle{usercode} = [rectangle,
                             %rounded corners,
                             minimum width=1.5cm,
                             minimum height=0.6cm,
                             text centered,
                             draw=black,
                             text width=1.5cm,
                             fill=green!20]
      \tikzstyle{arrow} = [thick,->,>=stealth]

      \begin{scope}[xshift=-3.5cm]
        \node at (0,1) {Standard method};
        \node (input) [io] {Initial state};
        \node (odesolver) [libcode, below of=input] {ODE Solver};
        \node (output) [io, below of=odesolver] {Final state};
        \node (rhs) [usercode, right of=odesolver, node distance=2.3cm]
          {$M\,\text{grad}\,H$};
        \draw [arrow] (input) -- (odesolver);
        \draw [arrow] (odesolver) -- (output);
        \draw [arrow] (odesolver) -- (rhs);
        \draw [arrow]  (rhs) -- (odesolver);
      \end{scope}

      \begin{scope}[xshift=3.5cm]
        \node at (0,1) {Proposed method};
        \node (input) [io] {Initial state};
        \node (odesolver) [libcode, below of=input] {ODE Solver};
        \node (output) [io, below of=odesolver] {Final state};
        \draw [arrow] (input) -- (odesolver);
        \draw [arrow] (odesolver) -- (output);
        \node (rhs) [libcode, fill=yellow!35, right of=odesolver,
                     node distance=2.5cm]
          {General purpose\\ $M\,\text{grad}\,f$ routine};
        \draw [arrow] (odesolver) -- (rhs);
        \draw [arrow]  (rhs) -- (odesolver);
        \node (h) [usercode, right of=rhs, node distance=2.5cm] {$H$ function};
        \draw [arrow] (rhs) -- (h);
        \draw [arrow] (h) -- (rhs);
      \end{scope}
    \end{tikzpicture}
  \caption{Flowcharts for the standard and the proposed methods. Round cornered
    boxes represent library or general purpose code; right angled boxes contain
    code for the specific calculation at hand, which the user must
  supply.\label{fig:a}}
\end{figure}

The procedure for the calculation of the partial derivatives of the Hamiltonian
(the "General purpose $M\text{grad} f$" block in the flowchart to the right in
Figure~\ref{fig:a}) can be implemented in many standard ways. For example, using
central difference formulas, we can write it in pseudocode as\\
\begin{algorithm}[H]
  \For{$1\leq i\leq N$}{
    $\partial_{q_i}H=\left(H(t,\; \tilde q, q_i+\delta q_i,\; \tilde{p})-
    H(t,\; \tilde q, q_i-\delta q_i,\; \tilde{p})\right)/(2\delta q_i)$\\
    $\partial_{p_i}H=\left(H(t,\; \tilde q,\; \tilde{p}, p_i+\delta p_i)-
    H(t,\; \tilde q,\; \tilde{p}, p_i-\delta p_i)\right)/(2\delta p_i) $
  } 
  \Return
  \{
    $\partial_{p_1}H,\; \partial_{p_2}H,\;\ldots, \partial_{p_N}H,\;
    -\partial_{q_1}H,\; -\partial_{q_2}H,\;\ldots, -\partial_{q_N}H$
  \}
  
\end{algorithm}
\noindent
where the notation $\tilde q$ above represents all the coordinates
$q_1, \ldots, q_N$, except the $i$-th coordinate $q_i$, and similarly for the
momenta ($\tilde p$).

%To make this discussion clearer, let us consider a specific problem, the
%classical harmonic oscillator~\cite{french:71}, and a specific ODE solver method,
%the explicit forward Euler method \cite{sir:2010}. The hamiltonian function for
%a classical oscillator with mass $m$ and restoring force constant $k$ is 
%\begin{equation}
%  H(q,p)=\frac{1}{2}kq^2+\frac{p^2}{2m}.
%\end{equation}
%Hamilton's equations~\eqref{eq:heqs} then read
%\begin{align}
%  \frac{dq}{dt}&=\frac{p}{m}&
%  \frac{dp}{dt}&=-kq,
%\end{align}
%which are already in the form of eq.~\eqref{eq:numheqs}. Given initial values
%$q_0$ and $p_0,$ successive values can be estimated at times $t_1$, $t_2$, \ldots
%by iterating the formulas
%\begin{align}
%  q_{i+1}&=q_{i}+\frac{p_i}{m}\delta t_i\label{eq:efe1}\\
%  p_{i+1}&=p_{i}-kq_i\delta t_i\label{eq:efe2},
%\end{align}
%where $\delta t_i=t_{i+1}-t_i$. It is now easy to write a  program that
%implements this method, and this the usual procedure. With this paper, we
%consider a method where explicit expressions for the hamiltonian's partial
%derivatives in the equations of motion are replaced by numerical estimates. In
%our toy problem, that amounts to rewriting eqs~\eqref{eq:efe1}
%and~\eqref{eq:efe2} by
%\begin{align*}
%  q_{i+1}&=q_{i}+\numap{\frac{\partial H}{\partial p}}_i \delta t_i\\
%  p_{i+1}&=p_{i}-\numap{\frac{\partial H}{\partial q}}_i\delta t_i,
%\end{align*}
%where $\numap{X}_i$ means \emph{numerical approximation of $X$ at time $t_i$.}
%In order to make numerical aproximations to the partial derivatives of a
%function, one must evaluate it several times; at the very least, two evaluations
%are needed. For example, using the second order central difference method, the
%derivative is estimated as
%\begin{equation}\label{eq:cdiff}
%  \frac{\partial\psi}{\partial x} = 
%    \frac{\psi(x+\delta x)-\psi(x-\delta t)}{2\delta x}+
%  o(\delta x)^2,
%\end{equation}
%leading to approximation formulas that, for our example problem, read 
%\begin{align}
%  \numap{\frac{\partial H}{\partial p}}_i &=
%  \frac{H(q_i, p_i+\delta p)-H(q_i, p_i-\delta p)}{2\delta p}\\
%  \numap{\frac{\partial H}{\partial q}}_i &=
%  \frac{H(q_i+\delta q, p_i)-H(q_i-\delta q, p_i)}{2\delta q},
%\end{align}
%where $\delta q$ and $\delta p$ are suitably chosen discretization parameters.
%Other approximations have higher accuracy, at the cost of extra evaluations of
%the function to be derived.

%This need for repeated evaluation of the system's hamiltonian reduces the
%efficiency of this method. However, programming a procedure for the hamiltonian
%function is usually much simpler then doing it for the set of its partial
%derivatives.  Also, adjusting the syntax and semantics of the ODE solver and the
%hamiltonian procedure is likewise much simpler than when the partial derivatives
%are coded.  For small systems, using the hamiltonian is much simpler and faster.
%
%
%
\section{Simple implementation in Python and an example}
We made a simple implementation of this method in Python, using standard
libraries like the numerical extensions package
NumPy~\cite{numpy:2011,numpy:2020} and the scientific library
SciPy~\cite{scipy}.  It consists of a simple function that calls the hamiltonian
function supplied by the user to evaluate, reorder and return its partial
derivatives, using central difference formulas. The step parameters $\delta q_i,
\delta p_i$ for the derivation formulas must be supplied by the user.  This
function is then given as an argument to SciPy's ODE solver \texttt{solve\_ivp}.
For greater simplicity of use, it is wrapped in a class, whose objects store
initialization details on creation, like the hamilton function, its parameters,
the number of degrees of freedom, the values of the discretization steps, etc.
Furthermore, these objects are defined as callable, returning the list of rhs of
hamilton's equations for given values of the dynamical variables $q_i, p_i$ and
time, so that they can be supplied to general ODE integrators just as if those
rhs functions had been explicitly programmed.

The whole class definition (including comments and python doc strings) fits in a
few dozen lines of code and is very easy to use. It is available for download as
free software at github~\cite{nevh:2020}.  The repository also includes a
\texttt{jupyter notebook} showcasing several example applications.  Here we
present a particularly simple application example, the one-dimensional harmonic
oscillator (also included in the repository).

\begin{lstlisting}[caption= {Numerical solution of Hamiltonian's equations for
the one-dimensional linear harmonic oscillator. Note that the user only supplies
code for the system's Hamilton function in lines 7-9. The remaining code lines
define initial state, physical and numerical parameters.},
numbers=left, stepnumber=1, label={lst:1dlho}, language=python
]
import numpy as np
from scipy.integrate import solve_ivp
import matplotlib.pyplot as plt
import nevh

# Hamiltonian function
def H(t, s, k, m):
    x, p = s
    return 0.5*k*x**2 + 0.5*p**2/m

# "Administrative" details
# Initial state: off equilibrium position, at rest
x0 = 1.0; p0 = 0.0; initial_state = [x0,p0]
# Integration steps parameters for partial derivatives dH/dq and dH/dp
ds = np.ones(2)*0.1
# Hamiltonian parameters. With k=4\pi^2, m=1, the period is 1
kc = 4 * np.pi**2; mc = 1.0
wc = (kc / mc)**0.5; Tc = 2 * np.pi / wc

# Define object that computes and returns the rhs of Hamilton's eqs
G = nevh.Hgrad(H, ds, k=kc, m=mc)
tmin = 0; tmax = 2 * Tc # two full oscillation periods
# Solve Hamilton's equations
trj = solve_ivp(G, [tmin, tmax], initial_state)
\end{lstlisting}
Scipy's ODE solver function \texttt{solve\_ivp} returns a composite structure
(named \texttt{trj} in Listing~\ref{lst:1dlho}) that stores information
regarding the numerical solution of the ODE. Namely, sampled values of the
coordinates and momenta at different times in the interval \texttt{[tmin, tmax]}
are stored at the arrays \texttt{trj.y[0]}, \texttt{trj.y[1]}, while the
sampling times are stored in \texttt{trj.t}. The sampling frequency can be
adjusted by the user, but here we just took the default behaviour of
\texttt{solve\_ivp}, which sets that frequency accordind to precision
requirements (that we also didn't bother to specify, because it wasn't necessary
for convergence). 


More relevant to the method here presented, the values of the discretization
steps used for the numerical computation of the partial derivatives of the
hamiltonian, stored in array \texttt{ds} in Listing~\ref{lst:1dlho}, must be
supplied by the user. In the example shown these values are irrelevant, because
the hamiltonian of the linear harmonic oscilator is a quadratic function of all
the coordinates and conjugate momenta and the central difference formulas
yield exact estimates of the derivatives of quadratic functions. However, in
other problems, the values of the discretizations steps must be carefully
considered.

Figure~\ref{fig:1dlho}
displays the graphs of coordinates and momenta of the numerical solution
obtained (discrete points), together with plots of the analytical solutions for
the same problem.
\begin{figure}[htb]
  \begin{center}
    \includegraphics[width=0.6\linewidth]{figs/1dlho.png}
  \end{center}
  \caption{Plot of position (\texttt{x}) and momentum (\texttt{p}) for the
  one-dimensional harmonic oscillator problem of Listing~\ref{lst:1dlho}.
  }\label{fig:1dlho}
\end{figure}
The correctness of the numerical approximation, at least for pedagogical
purposes only, is manifest.

Note that we do not claim that this method is more efficient than the standard
approach. Much to the contrary, in computationally intensive applications, the
effort involved in the numerical calculation of the partial derivatives of the
hamiltonian (for which several computations of the hamiltonian function itself
are needed), at every instant in the simulation, slows the calculaton very
significantly. What we do claim is that our method makes it very simple to set
up a practical simulation of small mechanical systems, because the user is
spared the trouble of deriving expressions for the partial derivatives of the
hamiltonian and writing subroutines that compute them, with the calling
signature that the particular ODE solver expects.


\section{Conclusion}


\section*{References}
\begin{thebibliography}{99}
  \bibitem{gol:1980} H. Goldstein, \textsl{Classical Mechanics.} Addison-Wesley
  (1980)
%
  \bibitem{fw:2003} 
    A.L.~Fetter, J.D.~Walecka,
    \textsl{Theoretical Mechanics of Particles and Continua.}
    Dover (2003)
%
  \bibitem{odepack}
  A.C. Hindmarsh, \textsl{ODEPACK, A Systematized Collection of ODE Solvers,} in
    Scientific Com\-puting, R. S. Stepleman et al. (eds.), North-Holland,
    Amsterdam, 1983 (vol. 1 of IMACS Transactions on Scientific Computation),
    pp. 55-64. 
%
  \bibitem{odeint}
  K. Ahnert and M. Mulansky, \textsl{Odeint - Solving Ordinary Differential
  Equations in C++,} AIP Conf. Proc. 1389, pp. 1586-1589 (2011);
doi:http://dx.doi.org/10.1063/1.3637934
%
  \bibitem{scipy}
  P. Virtanen et al, \textsl{SciPy 1.0: Fundamental Algorithms for Scientific
  Computing in Python}, Nature Methods 17, pp. 261--272 (2020)
%
  \bibitem{french:71} A.P.~French, \textsl{Vibration and Waves.} W.W. Norton \&
  Company (1971)
%
  \bibitem{sir:2010} S. \v{S}irca, M. Horvat, \textsl{Computational Methods in
  Physics.} Springer (2012)
%
  \bibitem{nr:2007} W.H.~Press, S.A.~Teukolsky, W.T.~Vetterling, B.P.~Flannery,
  \textsl{Numerical Recipes -- The Art of Scientific Computing,} 3rd Ed.
  Cambridge University Press (2007)
%
  \bibitem{numpy:2011} S.J.~van der Walt, S.C.~Colbert, G.~Varoquaux,
  \textsl{The NumPy array: a structure for efficient numerical computation,}
  Computing in Science \& Engineering, \textbf{13} no. 2, 22-30 (2011).
  doi: 10.1109/MCSE.2011.37
%
  \bibitem{numpy:2020} C.R.~Harris, K.J.~Millman, S.J.~van der Walt et. al.,
  \textsl{Array programming with NumPy,} Nature \textbf{585}, 357-362 (2020).
  https://doi.org/10.1038/s41586-020-2649-2
%
  \bibitem{nevh:2020} J.M.L.~Amoreira, L.J.M.~Amoreira,
  \texttt{https://github.com/ljmamoreira/nevh} (2020)

\end{thebibliography}
\end{document}
