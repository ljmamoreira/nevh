\documentclass{iopart}
\usepackage[utf8]{inputenc}
\usepackage[T1]{fontenc}
\usepackage[english]{babel}
\expandafter\let\csname equation*\endcsname\relax
\expandafter\let\csname endequation*\endcsname\relax
\usepackage{amsmath}
\newcommand{\pd}[2]{\frac{\partial#1}{\partial#2}}
\begin{document}
\title{Numerical Evolution from the Hamiltonian}
\author{José M L Amoreira$^1$ and Luís J M Amoreira$^2$}
\address{$^1$ Departamento de Física, Instituto Superior Técnico, Lisboa,
Portugal}
\address{$^2$ Departamento de Física, Universidade da Beira Interior, Covilhã,
Portugal}
\ead{amoreira@ubi.pt}
\begin{abstract}
  We propose a numerical method for the approximate calculation of the time
  evolution of point particle systems given only the system's hamiltonian
  function and initial condition. The numerical method both generates \emph{and}
  solves the equations of motion numerically. For demonstration purposes, a
  working im\-ple\-men\-ta\-tion (written in \texttt{python}) is described and
  applied to standard problems. The method may have some pedagogical merits but
  the numerical effort of generating the equations of motion makes it unsuitable
  for actual numerical solution of ``real'' problems with any but just a few
  degrees of freedom.
\end{abstract}
\noindent{\it Keywords\/}: Classical particle dynamics,
Hamilton's equations,
Numerical methods

\submitto{\EJP}
%\maketitle
\section{Introduction}
In the hamiltonian formulation of classical mechanics, the time evolutin of
physical systems is governed by Hamilton's equations which, for point point
particle systems, take the form
\begin{align}
  \dot q_i&=\pd{H}{p_i}&
  \dot p_i&=-\pd{H}{q_i},
\end{align}
for $i=1,\;\ldots,\;N$ where $N$ is the number of degrees of freedom of the
system (three times the number of particle for unconstrined, three dimensional
systems), $q_i$, $p_i$ are the canonical coordinates and conjugate momenta
respectively, and $H=H(t, q_i, p_i)$ is the hamiltonian function of the system
(for XXX systems, the hamiltonian function is simply the mechanical energy of
the system). For discrete systems, these equations are a set of first order
ordinary differential equations (ODE) in time, whose solutions for given initial
conditions are the trajectories foolowed by the system's particles.

In all but a small handfull of very well known problems, this system of ODEs
must be solved numerically. The usual approach in such cases is to obtain
explicit expressions for the partial derivatives of the hamiltonian function
(the right hand sides of Hamilton's equations) and to code those expressions in
functions which are then supplied to a standard ODE solver subroutine. In this
work, we propose an alternative numerical method, where the user only supplies
the hamiltonian function of the system, leaving its partial derivatives to be
computed numerically by the ODE solver.

This procedure is computationally more expensive, since many evaluations of the
hamiltonian function are needed in order to estimate its partial derivatives at
any moment in time, and so it really is not suitable to the analisys of complex
systems with many coupled degreess of freedom. However, for simple systems it is
very practical and, anyway, it is an interesting comcept in it own that we
haven't yet seen exposed anywhere.

\section{Numerical procedure}
\section{Simple implementation in \texttt{python} and examples}
\section{Conclusion}

\section*{References}
\begin{thebibliography}{99}
  \bibitem{gol:1980} H. Goldstein, \textsl{Classical Mechanics.} Addison-Wesley
  (1980)
  \bibitem{sir:2010} S. \v{S}irca, M. Horvat, \textsl{Computational Methods in
  Physics.} Springer (2012)
\end{thebibliography}
\end{document}
