\documentclass{iopart}
\usepackage[utf8]{inputenc}
\usepackage[T1]{fontenc}
\usepackage[english]{babel}
\begin{document}
\title{Numerical Evolution from the Hamiltonian}
\author{José M L Amoreira$^1$ and Luís J M Amoreira$^2$}
\address{$^1$ Departamento de Física, Instituto Superior Técnico, Lisboa,
Portugal}
\address{$^2$ Departamento de Física, Universidade da Beira Interior, Covilhã,
Portugal}
\ead{amoreira@ubi.pt}
\begin{abstract}
  We propose a numerical method for the approximate calculation of the time
  evolution of point particle systems given only the system's hamiltonian and
  initial condition. The numerical method both generates \emph{and} solves
  the equations of motion numerically. For demonstration purposes, a working
  im\-ple\-men\-ta\-tion (written in \texttt{python}) is described and applied to
  standard problems. The method may have some pedagogical merits but the
  numerical effort of generating the equations of motion makes it unsuitable for
  actual numerical solution of ``real'' problems with any but just a few degrees
  of freedom.
\end{abstract}
\submitto{\EJP}
\maketitle
\section{Introduction}

\end{document}
